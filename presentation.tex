\documentclass{beamer}

\usepackage{graphicx}
\usepackage{listings}
\usepackage{xcolor}
\usepackage{hyperref}
\usepackage{marvosym}
\usepackage{fontawesome}
\usepackage{tikz}
\usepackage[marvosym]{tikzsymbols}

\definecolor{commentgreen}{rgb}{0,0.6,0}

\lstset {
  language=C++,
  basicstyle=\footnotesize\ttfamily,
  columns=fullflexible,
  escapeinside={(*}{*)},
  keywordstyle=\color{blue},
  commentstyle=\color{commentgreen}
}

\hypersetup{
  colorlinks=true,
  linkcolor=blue,
  urlcolor=cyan,
}

\title{Introduction to Git}
\author{Cengiz Kandemir}
\date{\today}

\begin{document}

\frame{\titlepage}

\begin{frame}
  \frametitle{Who am I?}
  \begin{itemize}
  \item[] Currently: Software Engineer @ Sioux, working w/ Philips IGT
  \item[] Before:
    \begin{itemize}
    \item[] Software Engineer @ AirTies
    \item[] Research Assistant @ EMU (co-supervised by Cem Kalyoncu)
    \end{itemize}
  \end{itemize}
\end{frame}

\begin{frame}{Outline}
  \tableofcontents
\end{frame}

\section{What is Git and why do we need it?}
\begin{frame}
  \frametitle{What is Git?}
  \begin{quote}
    Git is software for \textbf{tracking changes} in any set of files, usually used for \textbf{coordinating work} among programmers collaboratively developing source code during software development. \\ \begin{flushright}Wikipedia\end{flushright}
  \end{quote}
\end{frame}

\begin{frame}
  \frametitle{Why do developers need a version control systems?}
  \begin{itemize}
  \item<1-> Synchronize changes made by two or more developers
  \item<2-> Undo/redo changes conveniently
  \item<3-> Bookmark (aka \textit{tag}) specific points in a repository's history
    \begin{itemize}
    \item<4-> Releases/important milestones
    \item<5-> Critical/risky changes
    \end{itemize}
  \item<6-> Improved traceability and visibility
    \begin{itemize}
    \item<7-> Traceability: Bugs can be narrowed down based on the locality of changes
    \item<8-> Visibility: Changes are visible to other stakeholders
      \begin{itemize}
      \item<8-> Improved collaboration
      \end{itemize}
    \end{itemize}
  \item<9-> Improved software quality
    \begin{itemize}
    \item<10-> Testing/Static analysis
    \item<11-> CI/CD pipelines
    \end{itemize}
  \end{itemize}
\end{frame}

\section{Basics of Git}
\begin{frame}
  \frametitle{Git loop}
  {\centering
    \begin{tikzpicture}[remember picture, overlay, node distance=4cm]
      \node [red, draw, circle, anchor=north, xshift=-0.5cm, yshift=-1cm] (clone) at (current page.north) {\color{black}Clone};
      \node [red, draw, circle, below of=clone, , xshift=-0.5cm] (pull) at (current page.north) {\color{black}Pull};
      \draw [blue,->] (clone) to (pull);
      \coordinate[below of=pull] (belowPull);
      \node [red, draw, circle, right of=belowPull, yshift=1.5cm, xshift=-2cm] (commit) {\color{black}Commit};
      \draw [blue,->] (pull) to [bend left] (commit);
      \node [red, draw, circle, left of=belowPull, yshift=1.5cm, xshift=2cm] (push) {\color{black}Push};
      \draw [blue,->] (push) to [bend left] (pull);
      \draw [blue,->] (commit) to [bend left] (push);
      \draw [blue,->] (commit.-90) arc (180:264+180:7.4mm) (commit);
    \end{tikzpicture}}
\end{frame}

\begin{frame}
  \frametitle{git-clone}
  \begin{itemize}
  \item<1-> Clones a git repository (\textit{.git} folder)
  \item<2-> A repository can be located \textit{\textbf{locally}} or \textit{\textbf{remotely}}
  \item<3->[] \textit{git clone} {\textless}\textit{path-to-repo}{\textgreater}
  \end{itemize}
\end{frame}

\begin{frame}
  \frametitle{git-clone cont.}
  {\centering
  \begin{tikzpicture}[remember picture, overlay, node distance=2cm]
    \node [] (remote) at (current page.center){\Large\faDatabase};
    \coordinate[above of=remote] (remoteCoord);
    \node [left of=remoteCoord] (devA) {\fontsize{45}{45}\tikzsymbolsuse{Gentsroom}};
    \draw [->] (remote) to [bend right] node [anchor=south west] {git clone} (devA);
  \end{tikzpicture}}
\end{frame}

\begin{frame}
  \frametitle{git-add \& git-commit}
  \begin{itemize}
  \item<1-> A commit is composed of two steps: staging and committing
  \item<2-> Staging is useful for compartmentalizing changes
  \item<3->[] \textit{git add} {\textless}\textit{list-of-files-to-be-staged}{\textgreater}
  \item<3->[] \textit{git stage} does the same
  \item<3->[] \textit{git commit}
  \item<4-> The state of repository after a commit is recorded and can always be returned to
  \item<5-> Committing a set of changes creates a ``bookmark``, identified by a \textit{\textbf{commit hash}}
  \item<6-> Many commands in Git require a commit hash as an input
  \end{itemize}
\end{frame}

\begin{frame}
  \frametitle{git-add \& git-commit cont.}
  {\centering
  \begin{tikzpicture}[remember picture, overlay, node distance=2cm]
    \node [] (remote) at (current page.center){\Large\faDatabase};
    \coordinate[above of=remote] (remoteCoord);
    \node [left of=remoteCoord] (devA) {\fontsize{45}{45}\tikzsymbolsuse{Gentsroom}};
    \draw [->] (remote) to [bend right] node [anchor=south west] {git clone} (devA);
    \draw [->] (devA.90) arc (0:264:4mm) node[pos=2, anchor=east] {git commit} (devA);
  \end{tikzpicture}}
\end{frame}

\begin{frame}
  \frametitle{git-push}
  \begin{itemize}
  \item<1-> Updates remote/server with the local changes
    \begin{itemize}
    \item<2-> Changes will be visible to everyone
    \item<3-> More robust to storage faults (if stored remotely)
    \end{itemize}
  \end{itemize}
\end{frame}

\begin{frame}
  \frametitle{git-push cont.}
  {\centering
  \begin{tikzpicture}[remember picture, overlay, node distance=2cm]
    \node [] (remote) at (current page.center){\Large\faDatabase};
    \coordinate[above of=remote] (remoteCoord);
    \node [left of=remoteCoord] (devA) {\fontsize{45}{45}\tikzsymbolsuse{Gentsroom}};
    \draw [->] (remote) to [bend right] node [anchor=south west] {git clone} (devA);
    \draw [->] (devA.90) arc (0:264:4mm) node[pos=2, anchor=east] {git commit} (devA);
    \draw [->] (devA) to [bend right] node [anchor=north east] {git push} (remote);
  \end{tikzpicture}}
\end{frame}

\begin{frame}
  \frametitle{git-fetch \& git-pull}
  \begin{itemize}
  \item<1-> \textit{git-fetch} fetches changes from the remote
    \begin{itemize}
    \item<2->[] \textit{git fetch}
    \end{itemize}
  \item<3-> \textit{git-pull} fetches changes from the remote and integrates the changes
    \begin{itemize}
    \item<4->[] \textit{git pull}
    \end{itemize}
  \end{itemize}
\end{frame}

\begin{frame}
  \frametitle{git-fetch \& git-pull cont.}
  {\centering
  \begin{tikzpicture}[remember picture, overlay, node distance=2cm]
    \node [] (remote) at (current page.center){\Large\faDatabase};
    \coordinate[above of=remote] (aboveRemote);
    \coordinate[below of=remote] (belowRemote);
    \node [left of=aboveRemote] (devA) {\fontsize{45}{45}\tikzsymbolsuse{Gentsroom}};
    \node [right of=belowRemote] (devB) {\fontsize{45}{45}\tikzsymbolsuse{Ladiesroom}};
    \draw [->] (remote) to [bend right] node [anchor=south west] {git pull} (devA);
    \draw [->] (devA.90) arc (0:264:4mm) node[pos=2, anchor=east] {git commit} (devA);
    \draw [->] (devB) to [bend left] node [anchor=north east] {git push} (remote);
  \end{tikzpicture}}
\end{frame}

\begin{frame}
  \frametitle{git-status \& .gitignore file}
  \begin{itemize}
  \item<1-> \textit{git-status} shows current state of the repository
    \begin{itemize}
    \item<2-> Staged and unstaged changes
    \item<2->[] \textit{git status} {\textless}\textit{a-path-in-repository}{\textgreater}
    \end{itemize}
  \item<3-> Compilation spits out a lot of intermediate files
    \begin{itemize}
    \item<4-> Executables/binaries
    \item<4-> Debug files
    \item<4-> Raw resources (text, image, etc.)
    \end{itemize}
  \item<5-> A way to ignore unwanted files: .gitignore
  \item<6-> Uses pattern matching to filter out files to be ignored
    \begin{itemize}
    \item<7-> filter build/Debug folder: ``build/'' or ``Debug/''
    \item<8-> filter individual files: simply write file name
    \item<9-> filter .txt files: ``*.txt''
    \end{itemize}
  \end{itemize}
\end{frame}

\begin{frame}
  \frametitle{git-log \& git-diff}
  \begin{itemize}
  \item<1-> \textit{git-log} shows the commit history
    \begin{itemize}
    \item<1->[] \textit{git log}
    \end{itemize}
  \item<1-> \textit{git-diff} shows the current changes (the ``diff'')
    \begin{itemize}
    \item<1->[] \textit{git diff}
    \end{itemize}
  \end{itemize}
\end{frame}

\begin{frame}
\centering{Time for a small demo.\\Let's create a git repo in Bitbucket from scratch.}
\end{frame}

\section{Best practices}
\begin{frame}[fragile]
  \frametitle{Commit atomic changes}
  \begin{lstlisting}
    std::string find_surname(int id)
    {
        std::map<int, std::string>& data = get_data();
        return data[id];
    }
  \end{lstlisting}
\end{frame}

\begin{frame}[fragile]
  \frametitle{Commit atomic changes}
   \begin{lstlisting}
    std::string (*\textit{\color{red}{get}}*)_surname(int id) // 1 - change the name
    {
        // 2 - fix a potential bug
        (*\textit{\color{red}{const}}*) std::map<int, std::string>(*\color{red}{\&}*) data = get_data();
        return data(*\textit{\color{red}{.at(id)}}*);
    }
  \end{lstlisting}
\end{frame}

\begin{frame}
  \frametitle{Commit early \& commit often}
  \begin{itemize}
  \item<1-> Related to atomic changes
  \item<2-> Atomic changes tend to be small
  \item<3-> Small changes can be committed frequently
  \item<4-> Inconvenient but useful. \textit{How?}
    \begin{itemize}
    \item<5-> Easily roll back on experimental changes
    \item<6-> Easier to develop two or more ``features'' in parallel
    \item<7-> Early feedback (extremely important)
    \end{itemize}
  \end{itemize}
\end{frame}

\begin{frame}
  \frametitle{Importance of commit messages}
  \begin{itemize}
  \item<1-> Why is this important?
    \begin{itemize}
    \item<2-> You may forget how your code works
    \item<3-> You will \textbf{likely} forget what you meant in the commit message
    \item<4-> Others will \textbf{most definitely not} understand what you mean 
    \end{itemize}
  \item<5-> How to improve?
    \begin{itemize}
    \item<6-> Be grammatically correct
    \item<7-> Do not be afraid of writing long texts when needed
    \item<8-> Express ``why'' (intent) and not ``how'' (most common problem)
      \begin{itemize}
      \item <8-> Do not repeat what the code already says
      \item <8-> Changes to code already show the ``how'' part
      \end{itemize}
    \end{itemize}
  \end{itemize}
\end{frame}

\section{Summary}
\begin{frame}
  \frametitle{Summary}
  \begin{itemize}
  \item<1-> Vital to software development for coordination and change tracking
  \item<2->[] \textit{git clone}
  \item<2->[] \textit{git add ...}
  \item<2->[] \textit{git commit}
  \item<2->[] \textit{git push}
  \item<3-> Commit small changes
  \item<4-> Express ``why'' in commit messages
  \end{itemize}
\end{frame}

\begin{frame}
  \frametitle{Some resources}
  \centering{
    git help {\textless}command-name{\textgreater}\\
    \href{https://git-scm.com/doc}{https://git-scm.com/doc}\\
    \href{https://gitexplorer.com}{https://gitexplorer.com}
  }
\end{frame}

\begin{frame}
  \frametitle{Closure}
  \centering{Thanks!\\Questions?}
\end{frame}

\end{document}
%%% Local Variables:
%%% mode: latex
%%% TeX-master: t
%%% End:
